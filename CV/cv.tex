\documentclass[ComputerScience]{vita}
\usepackage{fullpage}
\usepackage{times}
\usepackage{url}

\newcommand{\comment}[1]{}

\begin{document}
\name{\huge{Alex David Groce}}
%\businessAddress{School of Electrical Engineering and Computer Science\\ Oregon State University\\ 1148 Kelley Engineering Center\\ Corvallis, OR 97331-5501} 
%  \homeAddress {US Citizen}


\newcategory{Teaching}
\newcategory{Current Students}
\newcategory{Graduated Students}
\newcategory{Research Interests}
\newcategory{Selected Presentations}
\newkind{Topics}
\newcategory{Education}
\newcategory{Professional Activities and Service}
\newcategory{Books, Edited Volumes}
\newcategory{Refereed Conference and Workshop Publications}
%\newcategory{Refereed Conference Tool Publications}
%\newcategory{Refereed Workshop Publications}
\newcategory{Refereed Journal Publications}
\newcategory{Technical Reports}
\newcategory{Columns, Book Reviews, and Magazine Articles}
\newcategory{Invited Seminars}
\newcategory{Invited Talks and Panels}
\newcategory{Invited Papers}
\newcategory{Panel and Committee Service}
\newcategory{Funding}
\newkind{Expected}

\begin{vita}

%\begin{Expected Degrees}
%\end{Expected Degrees}

\begin{Education}
  \item Ph.D., {\bf Computer Science}, Carnegie Mellon University, March 2005\\ Thesis Title:  {\bf Error Explanation and Fault Localization with Distance Metrics}.\\Thesis Committee:  Edmund Clarke (chair), David Garlan, Reid Simmons, and Willem Visser

  \item B.S., {\bf Computer Science}, North Carolina State University, May 1999 (\emph{summa cum laude})
  \item B.S., {\bf Multidisciplinary Studies}, North Carolina State University, May 1999 (\emph{summa cum laude})
\item  \hspace{0.5in} Minors:  {\bf English Literature} and {\bf Science, Technology, and Society}
\end{Education}

\begin{Experience}
\item {\bf Summary:} I have over fifteen years of experience in
  research, development, and analysis of complex software systems.  My
  work focuses on testing, developing, specifying, and understanding
  critical distributed, embedded, aerospace, systems, and security
  software and software/hardware systems, in order to increase the
  reliability, security, and efficiency of these systems.  I
  contributed significantly to the implementation and design of
  verification and analysis tools used internationally for research,
  industrial application, and teaching, including NASA's Java
  PathFinder 2, CBMC, MAGIC, SyMP, JPL's LogScope, the Concurrency
  Workbench (NC), DeepState, and TSTL.  Recently I have been developing languages
  and systems tools to make software testing and development of secure
  code easier for real-world users without a research background.
  I have been PI or co-PI on externally funded grants totalling over
  \$9.1M (\$1.2M my share), and have authored or co-authored more than
  70 publications, primarily in ACM, IEEE, and top verification and
  formal methods venues, with over 3,600 citations, an h-index of 30,
  and an i10-index of 62, according to Google Scholar.  I served on
  the program committees for ICSE, ASE, and ISSTA multiple times in
  recent years, write a regular column for ACM SIGSOFT Software
  Engineering Notes, and am IEEE Software's associate editor for
  Software Testing.  I make 500-1,500 GitHub commits in a typical
  year, and have reported over 100 bugs in widely used open source
  software systems as part of my research.

\item {\bf 2/2019--8/2019 $\cdot$ Security Engineer, Trail of Bits
    (Sabbatical):} Explored novel approaches to analyzing smart
  contracts and blockchain implementations, combining static and dynamic
  methods.

\item {\bf 1/2017--present $\cdot$ Associate Professor with Tenure, School of Informatics, Computing, and Cyber Systems, Northern Arizona University:} Explored methods for bringing advanced automated testing techniques into development practice, especially for Python and other languages used widely in bioinformatics and other scientific applications.  Continued to advance state-of-the-art in software testing.

  \item {\bf 9/2015--12/2016 $\cdot$ Associate Professor with Tenure, School of Electrical Engineering and Computer Science, Oregon State University:} Continued investigations of core problems in software and software/hardware systems understanding and correctness, including lightweight tools for automated testing and understanding by non-experts.  Began investigations of self-repairing and evolving systems for software longevity.  Co-chaired undergraduate curriculum committee.

  \item {\bf 6/2009--9/2015 $\cdot$ Assistant Professor, School of
    Electrical Engineering and Computer Science, Oregon State
    University:} Initiated investigation of methods and metrics for
    end-user testing of machine learning systems.  Continued work
    on unified approaches to testing, with new techniques applied to finding flaws in widely used production-quality C compilers, JavaScript engines, and embedded file systems.  Mentored graduate
    students, taught graduate and undergraduate classes on software
    engineering and verification, and served on graduate and undergraduate curriculum committees and the hiring committee.

  \item {\bf 2/2011--5/2015 $\cdot$ Consultant, Aries Design Automation, LLC:} Provided expertise on verification, testing, and analysis for SBIR and other projects and proposals.  Main consultant on NIST SBIR project ``Using Automated Abstractions to Classify System States for Software Health Monitoring,'' aimed at improving systems engineering of health monitoring systems via code analysis and machine learning.

  \item {\bf 4/2008--6/2008, 4/2009-6/2009 $\cdot$ Lecturer in
    Computer Science, part-time, California Institute of Technology:}
    Taught CS 119, Reliable Software: Testing and Monitoring.  Used
    research and JPL flight system experiences to introduce
    state-of-the-art techniques and practical methods for testing and
    verification to students, with a focus on automated approaches to
    executing software in order to find faults.

  \item {\bf 4/2005--6/2009 $\cdot$ Laboratory for Reliable Software,
    Jet Propulsion Laboratory:} Led test automation development and
    design, Mars Science Laboratory (Curiosity rover) Flight Software
    Internal Test Team.  Introduced new techniques for exploiting
    traces in static analysis of programs, integrated model checking
    and dynamic analysis, developed a successful random testing
    approach for mission file systems, and contributed to modeling and
    verification of Dawn mission launch sequence and fault protection.
    Led file system acceptance testing for a NASA Discovery class
    mission; led model checking and random testing efforts for Mars
    Science Laboratory (Curiosity) file storage modules; contributed
    to design for file systems used to store images, science products,
    and telemetry during spaceflight missions.  Served on design and
    code review panels for flight software systems and hardware
    drivers.  Worked with systems engineers to develop methods for
    specifying, generating, and understanding logs of complex
    spacecraft software and hardware activity.

  \item {\bf 8/1999--3/2005 $\cdot$ Doctoral student, Carnegie Mellon
    University:} Invented methods for error explanation and fault
    isolation using distance metrics, applied to aerospace, security, and
    micro-kernel code.  Invented and implemented novel approaches for
    counterexample guided abstraction refinement and heuristic search
    guided model checking.  Implemented Athena proof system for
    security protocols, devised language and type system for encoding
    security protocols in the SyMP tool.  Enriched teaching by
    incorporating research ideas into instruction, assignments, and
    evaluation in undergraduate classes.

  \item {\bf 5/2002--8/2002 $\cdot$  Summer Student Research Program, RIACS (Research Institute for Advanced Computer Science)/NASA Ames Research Center, Robust Software Engineering group:}  Invented and implemented methods for error explanation, using model checking counterexamples to provide automatic feedback about the causes and location of errors in complex systems.

  \item {\bf 5/2001--8/2001 $\cdot$  Summer Student Research Program, RIACS/NASA Ames Research Center, Robust Software Engineering group:}  Invented and experimented with novel (and successful) heuristics for model checking Java programs; implemented heuristic search in the Java PathFinder 2 model checker.

  \item {\bf 5/2000--8/2000 $\cdot$  Research intern, Bell Laboratories (Murray Hill):}  Implemented black box checking algorithm (model checking for an unknown model using finite-state machine learning algorithms) and investigated theoretical aspects and applications of the algorithm to software model checking.

  \item {\bf 5/1999--8/1999 $\cdot$ Summer research assistant, SUNY Stony Brook:}  Continued work from the previous summer.

  \item {\bf 5/1998--8/1998 $\cdot$ Summer research assistant, North Carolina State University:}   Implemented a model checker based on Alternating B\"uchi Tableau Automata and developed logical optimizations for ABTAs.

\end{Experience}

\begin{Research Interests}
\item \emph{Designing}, \emph{specifying}, \emph{coding},
  \emph{testing}, \emph{verifying}, \emph{understanding}, and
  \emph{debugging} complex computer systems.   My research has combined testing, static analysis, formal methods, programming
  languages, and machine learning approaches as required.  My current
  focus is on techniques for testing critical systems including
  compilers, web browsers, file systems, libraries and infrastructure,
  and more generally, all layers of modern computer system
  environments.  In particular, I am interested in bridging the gap
  between ``security'' approaches to automated testing (e.g. fuzzers
  like AFL and libFuzzer, binary analysis tools) and those more rooted in the software
  testing community; since ``bugs are bugs'' there is no reason for a
  deep divide in aggressive testing practice or theory.
  In the spirit of Henry Petroski's proposal that progress in
  engineering arises from understanding failures, I believe that a
  deeper understanding of {\bf bugs} is essential to better
  \emph{software and systems engineering} and better engineering
  education.
\begin{Topics}
\item {\bf Software testing:} I am exploring the effectiveness and relative ease of (randomized) testing, and the relationship between testing, runtime verification or dynamic analysis, and model checking using unsound abstractions --- including shared models and frameworks for testing and model checking and strategies based on constraint-solving and machine learning.  I am aiding core developers of the Linux kernel to use mutation analysis to improve kernel systems testing methods, and to verify critical algorithms. 
\item {\bf Software model checking:}  I continue to investigate the use of SAT and SMT solvers for bounded model checking (CBMC), predicate abstraction-based approaches (MAGIC, SATABS), and explicit-state exploration with SPIN and Java PathFinder. 
%\item {\bf Error explanation:}   I developed automatic tools based on model checking for \emph{explaining} and \emph{localizing} errors in systems.  Understanding and correcting errors can be as difficult and costly as implementation or design, and better methods and algorithms are critical to improving systems reliability.
\item{\bf Understanding complex program executions:} I am working with scientists at the United States Forest Service to analyze complex models used to predict climate change impacts, with techniques that should also apply to analyzing model checking and test system executions.
\item{\bf Educational use of analysis tools:}  I am interested in incorporating mature, robust ``research'' tools for system design, debugging, and verification (model checking, random testing, static analysis, delta-debugging, etc.) into engineering education:  I believe such tools not only make for better engineering practice, but make the learning experience more rewarding and interesting to students.
\comment{
\item {\bf Coverage for model checking:} I am interested in coverage metrics and certification methods for model checkers.  Traditional and cutting-edge measures from the world of testing can be applied to model checking runs, and proof-based techniques for certification have recently appeared.  However, the general problem remains:  after verification, how much confidence should we have in in the correctness of a program?}
\end{Topics}
\end{Research Interests}


\begin{Honors}
\item ACM 2017 {\bf Senior Member}
\item ACM/IEEE International Conference on Software Engineering 2016 {\bf Distinguished Poster Award}
  \item IEEE International Conference on Software Testing, Verification and Validation (ICST) 2014 {\bf Best Paper Award}
  \item {\bf NASA Software of the Year Award} for Mars Science Laboratory Flight Software, 2013 (MSL Flight Software team)
  \item {\bf NASA Space Act Award} for LogScope Software, 2011
  \item National Science Foundation Faculty {\bf Early Career Development (CAREER) Program Award}, 2011
  \item {\bf JPL Mariner Award} for LogScope Testing Software, 2009
  \item {\bf JPL Spot Award} (for Multi Mission System Architecture Platform
  (MSAP) File System Testing), 2006 \item ACM/IEEE International Conference on Software Engineering 2003 {\bf ACM SIGSOFT
  Distinguished Paper Award} \item {\bf NASA ``Engineering Innovation''
  Turning Goals Into Reality (TGIR) Award} 2003 (Java PathFinder team)
  \item {\bf National Science Foundation Graduate Fellowship} \item {\bf NCSU
  Class of 1999 College of Humanities and Social Sciences Scholar}
  (valedictorian for CHASS) \item {\bf Phi Beta Kappa}
\end{Honors}


\begin{Funding}
\item ``Interfaces, Models, and Monitoring for Resource-aware Transformations that Augment the Lifecycle of Systems (IMMoRTALS)'', PIs: Matt Gillen (BBN), Doug Schmidt (Vanderbilt), Eric Walkingshaw (Oregon State University), Heng Yin (Syracuse), Co-PIs: Alex Groce, Jules White (Vanderbilt), Jacob Staples (Securboration), DARPA BRASS (Building Resource Adaptive Software Systems), BAA-15-36, \$464,625, \$1.6M total Oregon State University budget, (total project budget \$7.7M), October 2015-September 2019.
\item ``Advanced Tools for Effective Automated Test Generation'', PIs: Miroslav Velev (Aries Design Automation, LLC), Alex Groce, \emph{NASA Small Business Technology Transfer Phase I} T11.01-9878, \$52,037 (total budget \$125,000), July 2015-2016.
\item ``Explorations of Testing in the Cloud'', PI: Alex Groce, Amazon Web Services in Education Grants, \$10,000, January 2015-2017.
\item ``II-EN: Software Tools for Monte-Carlo Optimization'', PI: Alan Fern, Co-PIs: Alex Groce, Sinisa Todorovic, Prasad Tadepalli, Thomas Dietterich, \emph{National Science Foundation} CNS-1406049, \$442,366, October 2014-September 2017 (infrastructure development for cloud-based optimization tools with ML, graphics, and testing applications; includes \$12,000 dedicated compute time for software testing research).
\item ``Diversity and Feedback in Random Testing for Systems Software'', PIs: Alex Groce, John Regehr (University of Utah), \emph{National Science Foundation} CCF-1217824, \$242,244 (total budget \$491,280), September 2012-2015.
\item ``CAREER: Integrating Automated Software Testing Methods'', PI: Alex Groce, \emph{National Science Foundation} CCF-1054876, \$400,000, September 2011-2016.
\end{Funding}

\begin{Books, Edited Volumes}

\item
{\bf Alex Groce} and Stefan Leue (eds).
\newblock Proceedings of the 2nd International Workshop on 
Causal Reasoning for Embedded and safety-critical Systems Technologies.
\newblock Electronic Proceedings in Theoretical Computer Science, Volume 259, October 2017.

\item
{\bf Alex Groce} and Madanlal Musuvathi (eds).
\newblock Model Checking Software: Proceedings of the 18th International SPIN Workshop.
\newblock Springer-Verlag, LNCS 6823, 2011. 

\end{Books, Edited Volumes}

\begin{Refereed Journal Publications}
\item
{\bf Alex Groce}, Iftekhar Ahmed, Carlos Jensen, Paul E. McKenney, and
Josie Holmes.
\newblock How Verified (or Tested) is My Code? Falsification-Driven Verification and Testing.
\newblock \emph{Automated Software Engineering Journal}, 25(4):
917-960, December 2018.

\item
Josie Holmes, {\bf Alex Groce}, Jervis Pinto, Pranjal Mittal, Pooria Azimi, Kevin Kellar, and James O'Brien.
\newblock TSTL: the Template Scripting Testing Language.
\newblock \emph{International Journal on Software Tools for Technology Transfer}, 20(1):57-78, February 2018.


\item
Rahul Gopinath, Iftekhar Ahmed, Mohammad Amin Alipour, Carlos Jensen, and {\bf Alex Groce}.
\newblock Mutation Reduction Strategies Considered Harmful.
\newblock \emph{IEEE Transactions on Reliability}, 66(3): 854-874, September 2017.


\item
Rahul Gopinath, Iftekhar Ahmed, Mohammad Amin Alipour, Carlos Jensen, and {\bf Alex Groce}.
\newblock Does Choice of Mutation Tool Matter?
\newblock \emph{Software Quality Journal}, 25(3):871-920, September 2017.

\item
{\bf Alex Groce}, Mohammad Amin Alipour, Chaoqiang Zhang, Yang Chen, and John Regehr.
\newblock Cause Reduction: Delta Debugging, Even Without Bugs.
\newblock \emph{Journal of Software Testing, Verification and Reliability}, 26(1):40-68, January 2016.

\item
Milos Gligoric, {\bf Alex Groce}, Chaoqiang Zhang, Rohan Sharma, Mohammad Amin Alipour, and Darko Marinov.
\newblock Guidelines for Coverage-Based Comparisons of Non-Adequate Test Suites.
\newblock \emph{ACM Transactions on Software Engineering and Methodology}, 24(4):4-37, August 2015.

\item
{\bf Alex Groce}, Klaus Havelund, Gerard Holzmann, Rajeev Joshi, and Ru-Gang Xu.
\newblock Establishing Flight Software Reliability: Testing, Model Checking, Constraint-Solving, Monitoring and Learning.
\newblock \emph{Annals of Mathematics and Artificial Intelligence}, 70(4):315-348, April 2014.

\item
{\bf Alex Groce}, Todd Kulesza, Chaoqiang Zhang, Shalini Shamasunder, Margaret Burnett, Weng-Keen Wong, Simone Stumpf, Shubhomoy Das, Amber Shinsel, Forrest Bice, and Kevin McIntosh.
\newblock You Are the Only Possible Oracle: Effective Test Selection for End Users of Interactive Machine Learning Systems.
\newblock \emph{IEEE Transactions on Software Engineering}, 40(3):307-323, March 2014.

\item 
Gerard Holzmann, Rajeev Joshi, and {\bf Alex Groce}.
\newblock Swarm Verification Techniques.
\newblock \emph{IEEE Transactions on Software Engineering}, 37(6):845-857, November 2011.

\item
Howard Barringer, {\bf Alex Groce}, Klaus Havelund, and Margaret Smith.
\newblock Formal Analysis of Log Files.
\newblock \emph{Journal of Aerospace Computing, Information, and Communication}, 7(11):365-390, December 2010.

\item Gerard Holzmann, Rajeev Joshi, and {\bf Alex Groce}.
\newblock Model Driven Code Checking.
\newblock \emph{Automated Software Engineering Journal}, 15(3-4):283-297, December 2008.

\item
{\bf Alex Groce} and Rajeev Joshi.
\newblock Exploiting Traces in Static Program Analysis: Better Model Checking through \emph{printf}s.
\newblock \emph{International Journal on Software Tools for Technology Transfer}, 10(2):131-144, March 2008.

\item
{\bf Alex Groce}, Doron Peled, and Mihalis Yannakakis.
\newblock Adaptive Model Checking.
\newblock \emph{Logic Journal of the IGPL}, 14(5):729-744, October 2006.

\item
{\bf Alex Groce}, Sagar Chaki, Daniel Kroening, and Ofer Strichman.
\newblock Error Explanation with Distance Metrics.
\newblock \emph{International Journal on Software Tools for Technology Transfer}, 8(3):229-247, June 2006.

\item
{\bf Alex Groce} and Willem Visser.
\newblock Heuristics for Model Checking Java Programs.
\newblock \emph{International Journal on Software Tools for Technology Transfer}, 6(4):260-276, August 2004.


\item
Sagar Chaki, Edmund Clarke, {\bf Alex Groce}, Joel Ouaknine, Ofer Strichman, and Karen Yorav.
\newblock Efficient Verification of Sequential and Concurrent C Programs.
\newblock \emph{Formal Methods in System Design, Special Issue on Software Model Checking}, 25(2-3):129-166, September-November 2004.

\item
Sagar Chaki, Edmund Clarke, {\bf Alex Groce}, Somesh Jha, and Helmut Veith.
\newblock Modular Verification of Software Components in C.
\newblock \emph{IEEE Transactions on Software Engineering}, 30(6):388-402, June 2004.

\end{Refereed Journal Publications}

\begin{Refereed Conference and Workshop Publications}
\item Josselin Feist, Gustavo Grieco, and {\bf Alex Groce}. 
\newblock Slither: A Static Analysis Framework for Smart Contracts. 
\newblock \emph{International Workshop on Emerging Trends in Software
  Engineering for Blockchain}, accepted for publication, Montreal,
Canada, May 2019.
  
\item Josie Holmes and {\bf Alex Groce}. 
\newblock Causal Distance-Metric-Based Assistance for Debugging After 
Compiler Fuzzing. 
\newblock \emph{IEEE International Symposium on Software Reliability 
  Engineering}, pages 166-177, Memphis, Tennessee, October 
2018 (acceptance rate 24\%), {\bf invited for journal submission to STVR}.

\item Arpit Christi, Matthew Olson, Mohammad Amin Alipour, and {\bf Alex Groce}. 
\newblock Reduce Before You Localize: Delta-Debugging and Spectrum-Based Fault Localization. 
\newblock \emph{IEEE International Workshop on Debugging and Repair},
pages 184-191, Memphis, Tennessee, October  2018.

\item Peter Goodman, Gustavo Grieco,  and {\bf Alex Groce}. 
\newblock Tutorial: DeepState: Bringing Vulnerability Detection Tools 
into the Development Cycle. 
\newblock \emph{IEEE Cybersecurity Development Conference (SecDev)},
pages 130-131, Boston, Massachusetts, September-October 2018 (tutorials track, 
acceptance rate 55\%). 

\item Arpit Christi and {\bf Alex Groce}.
\newblock Target Selection for Test-Based Resource Adaptation.
\newblock \emph{IEEE International Conference on Software Quality,
  Reliability, and Security}, pages 458-469, 
Lisbon, Portugal, July 2018 (acceptance rate 19\%).

\item {\bf Alex Groce},  Josie Holmes, Darko Marinov, August Shi, and Lingming Zhang.
\newblock An Extensible, Regular-Expression-Based Tool for Multi-Language Mutant Generation.
\newblock \emph{ACM/IEEE International Conference on Software
  Engineering}, pages 25-28, Gothenburg, Sweden, May-June 2018 (demonstrations track).

\item Peter Goodman and {\bf Alex Groce}.
\newblock DeepState: Symbolic Unit Testing for C and C++.
\newblock NDSS Workshop on Binary Analysis Research, San Diego, California, February 2018.

\item {\bf Alex Groce} and Josie Holmes.
\newblock Provenance and Pseudo-Provenance for Seeded Learning-Based Automated Test Generation.
\newblock \emph{NIPS 2017 Interpretable ML Symposium}, Long Beach, California, December 2017.

\item Arpit Christi, {\bf Alex Groce}, and Rahul Gopinath.
\newblock Resource Adaptation via Test-Based Software Minimization.
\newblock \emph{IEEE International Conference on Self-Adaptive and Self-Organizing Systems}, pages 61-70, Tucson, Arizona, September 2017 (acceptance rate 21\%).

\item {\bf Alex Groce}, Josie Holmes, and Kevin Kellar.
\newblock One Test to Rule Them All.
\newblock \emph{ACM International Symposium on Software Testing and Analysis}, pages 1-11, Santa Barbara, California, July 2017 (acceptance rate 26\%).

\item {\bf Alex Groce}, Paul Flikkema, and Josie Holmes.
\newblock Towards Automated Composition of Heterogeneous Tests for Cyber-Physical Systems.
\newblock \emph{Workshop on Testing Embedded and Cyber-Physical Systems}, pages 12-15, Santa Barbara, California, July 2017.

\item Josie Holmes and {\bf Alex Groce}.
\newblock A Suite of Tools for Making Effective Use of Automatically Generated Tests.
\newblock \emph{ACM International Symposium on Software Testing and Analysis}, pages 356-359, Santa Barbara, California, July 2017 (Tools and Demonstrations track).

\item Rahul Gopinath, Carlos Jensen, and {\bf Alex Groce}.
\newblock The Theory of Composite Faults.
\newblock \emph{IEEE International Conference on Software Testing, Verification and Validation}, pages 47-57, Tokyo, Japan, March 2017 (acceptance rate 27\%).

\item Iftekhar Ahmed, Carlos Jensen, {\bf Alex Groce}, and Paul McKenney.
\newblock Applying Mutation Analysis on Kernel Test Suites: An Experience Report.
\newblock \emph{International Workshop on Mutation Analysis}, pages 110-115, Tokyo, Japan, March 2017.

\item Iftekhar Ahmed, Rahul Gopinath, Caius Brindescu, {\bf Alex Groce}, and Carlos Jensen.
\newblock Can Testedness be Effectively Measured?
\newblock \emph{ACM SIGSOFT International Symposium on the Foundations of Software Engineering},  pages 547-558, Seattle, Washington, November 2016 (acceptance rate 27\%).

\item Josie Holmes, {\bf Alex Groce}, and Mohammad Amin Alipour.
\newblock Mitigating (and Exploiting) Test Reduction Slippage.
\newblock \emph{Workshop on Automated Software Testing}, pages 66-69, Seattle, Washington, November 2016.

\item Ali Aburas and {\bf Alex Groce}.
\newblock A Method Dependence Relations Guided Genetic Algorithm.
\newblock \emph{International Symposium on Search-Based Software Engineering}, pages 267-273, Raleigh, North Carolina, October 2016 (short paper track).

\item Mohammad Amin Alipour, August Shi, Rahul Gopinath, Darko Marinov, and {\bf Alex Groce}.
\newblock Evaluating Non-Adequate Test-Case Reduction.
\newblock \emph{IEEE/ACM International Conference on Automated Software Engineering}, pages 16-26, Singapore, Singapore, September 2016 (acceptance rate 20\%).

\item Mohammad Amin Alipour, {\bf Alex Groce}, Rahul Gopinath, and Arpit Christi.
\newblock Generating Focused Random Tests Using Directed Swarm Testing.
\newblock \emph{ACM International Symposium on Software Testing and Analysis}, pages 70-81, Saarbrucken, Germany, July 2016 (acceptance rate 25\%).

\item Pranjal Mittal and {\bf Alex Groce}.
\newblock Poster: TSTL: A Little Language for Automated Testing Written in Python.
\newblock \emph{PyCon 2016}, Portland, Oregon, May-June 2016.

\item Rahul Gopinath, Amin Alipour, Iftekhar Ahmed, Carlos Jensen, and {\bf Alex Groce}.
\newblock On the Limits of Mutation Reduction Strategies.
\newblock \emph{ACM/IEEE International Conference on Software Engineering}, pages 511-522, Austin, Texas, May 2016 (acceptance rate 19\%).

\item Rahul Gopinath, Carlos Jensen, and {\bf Alex Groce}.
\newblock Poster: Topsy-Turvy: A Smarter and Faster Parallelization of Mutation Analysis.
\newblock \emph{ACM/IEEE International Conference on Software Engineering}, pages 740-743, Austin, Texas, May 2016 (poster track, acceptance rate 58\%), {\bf Distinguished Poster Award}.

\item Rahul Gopinath, Mohammad Amin Alipour, Iftekhar Ahmed, Carlos Jensen, and {\bf Alex Groce}.
\newblock Measuring Effectiveness of Mutant Sets.
\newblock \emph{International Workshop on Mutation Analysis}, pages 132-141, Chicago, Illinois, April 2016.

\item {\bf Alex Groce}, Iftekhar Ahmed, Carlos Jensen, and Paul E. McKenney.
\newblock How Verified is My Code? Falsification-Driven Verification.
\newblock \emph{IEEE/ACM International Conference on Automated Software Engineering}, pages 737-748, Lincoln, Nebraska, November 2015 (acceptance rate 21\%), {\bf invited for journal submission to ASE}.

\item Rahul Gopinath, Mohammad Amin Alipour, Iftekhar Ahmed, Carlos Jensen, and {\bf Alex Groce}.
\newblock How Hard Does Mutation Analysis Have to Be, Anyway?
\newblock \emph{IEEE International Symposium on Software Reliability Engineering}, pages 216-227, Gaithersburg, Maryland, November 2015 (acceptance rate 32\%).


\item {\bf Alex Groce}, Jervis Pinto, Pooria Azimi, and Pranjal Mittal.
\newblock TSTL: A Language and Tool for Testing (Demo).
\newblock \emph{ACM International Symposium on Software Testing and Analysis}, pages 414-417, Baltimore, Maryland, July 2015 (Tools and Demonstrations track). 

\item {\bf Alex Groce} and Jervis Pinto.
\newblock A Little Language for Testing.
\newblock \emph{NASA Formal Methods Symposium}, pages 204-218, Pasadena, California, April 2015 (acceptance rate 31\%).

\item Yuanli Pei, Arpit Christi, Xiaoli Fern, {\bf Alex Groce}, and Weng-Keen Wong.
\newblock Taming a Fuzzer Using Delta Debugging Trails.
\newblock \emph{International Workshop on Software Mining}, Shenzhen, China, December 2014.

\item Rahul Gopinath, Carlos Jensen, and {\bf Alex Groce}.
\newblock Mutations: How Close are they to Real Faults?
\newblock \emph{IEEE International Symposium on Software Reliability Engineering}, pages 189-200, Naples, Italy, November 2014 (acceptance rate 25\%).

\item
{\bf Alex Groce}, Mohammad Amin Alipour, and Rahul Gopinath.
\newblock Coverage and Its Discontents.
\newblock \emph{ACM Symposium on New Ideas in Programming and Reflections on Software, Onward! Essays, part of SPLASH (ACM SIGPLAN Conference on Systems, Programming, Languages and Applications: Software for Humanity)}, pages 255-268, Portland, Oregon, October 2014.

\item
Ali Aburas and {\bf Alex Groce}.
\newblock An Improved Memetic Algorithm with Method Dependence Relations (MAMDR).
\newblock \emph{International Conference on Quality Software}, pages 11-20, Dallas, Texas, October 2014 (acceptance rate 26\%).

\item
Chaoqiang Zhang, {\bf Alex Groce}, and Mohammad Amin Alipour.
\newblock Using Test Case Reduction and Prioritization to Improve Symbolic Execution.
\newblock \emph{ACM International Symposium on Software Testing and Analysis}, pages 60-70, San Jose, California, July 2014 (acceptance rate 28\%).

\item
Duc Le, Mohammad Amin Alipour, Rahul Gopinath, and {\bf Alex Groce}.
\newblock MuCheck: an Extensible Tool for Mutation Testing of Haskell Programs.
\newblock \emph{ACM International Symposium on Software Testing and Analysis}, pages 429-432, San Jose, California, July 2014 (Tools and Demonstration track). 

\item
Rahul Gopinath, Carlos Jensen, and {\bf Alex Groce}.
\newblock Code Coverage for Suite Evaluation by Developers.
\newblock \emph{ACM/IEEE International Conference on Software Engineering}, pages 72-82, Hyderabad, India, May-June 2014 (acceptance rate 20\%). 

\item
{\bf Alex Groce}, Mohammad Amin Alipour, Chaoqiang Zhang, Yang Chen, and John Regehr.
\newblock Cause Reduction for Quick Testing.
\newblock \emph {IEEE International Conference on Software Testing, Verification and Validation}, pages 243-252, Cleveland, Ohio, March-April 2014 (acceptance rate 28\%), {\bf Best Paper Award, invited for journal submission to STVR}.

\item
Amin Alipour, {\bf Alex Groce}, Chaoqiang Zhang, Anahita Sanadaji, and Gokul Caushik.
\newblock Finding Model-Checkable Needles in Large Source Code Haystacks: Modular Bug-Finding via Static Analysis and Dynamic Invariant Discovery.
\newblock \emph{International Workshop on Constraints in Formal Verification}, San Jose, California, November 2013.

\item {\bf Alex Groce}, Chaoqiang Zhang, Mohammad Amin Alipour, Eric Eide, Yang Chen, and John Regehr.
\newblock Help, Help, I'm Being Suppressed! The Significance of Suppressors in Software Testing.
\newblock \emph{IEEE International Symposium on Software Reliability Engineering}, pages 390-399, Pasadena, California, November 2013 (acceptance rate 35\%).

\item Milos Gligoric, {\bf Alex Groce}, Chaoqiang Zhang, Rohan Sharma, Amin Alipour, and Darko Marinov.
\newblock Comparing Non-adequate Test Suites using Coverage Criteria.
\newblock \emph{ACM International Symposium on Software Testing and Analysis}, pages 302-313, Lugano, Switzerland, July 2013 (acceptance rate 26\%), {\bf invited for journal submission to ACM TOSEM}.

\item Yang Chen, {\bf Alex Groce}, Chaoqiang Zhang, Weng-Keen Wong, Xiaoli Fern, Eric Eide, and John Regehr.
\newblock Taming Compiler Fuzzers.
\newblock \emph{ACM SIGPLAN Conference on Programming Language Design and Implementation}, pages 197-208, Seattle, Washington, June 2013 (acceptance rate 17\%).

\item {\bf Alex Groce}, Alan Fern, Jervis Pinto, Tim Bauer, Mohammad Amin Alipour, Martin Erwig, and Camden Lopez.
\newblock Lightweight Automated Testing with Adaptation-Based Programming.
\newblock \emph{IEEE International Symposium on Software Reliability Engineering}, pages 161-170, Dallas, Texas, November 2012 (acceptance rate 30\%).

\item Mohammad Amin Alipour and {\bf Alex Groce}.
\newblock Extended Program Invariants: Applications in Testing and Fault Localization.
\newblock \emph{International Workshop on Dynamic Analysis}, pages 7-11, Minneapolis, Minnesota, July 2012.

\item {\bf Alex Groce} and Martin Erwig.
\newblock Finding Common Ground: Choose, Assert, and Assume.
\newblock \emph{International Workshop on Dynamic Analysis}, pages 12-17, Minneapolis, Minnesota, July 2012.

\item {\bf Alex Groce}, Chaoqiang Zhang, Eric Eide, Yang Chen, and John Regehr.
\newblock Swarm Testing.
\newblock \emph{ACM International Symposium on Software Testing and Analysis}, pages 78-88, Minneapolis, Minnesota, July 2012 (acceptance rate 29\%).

\item {\bf Alex Groce}.
\newblock Coverage Rewarded: Test Input Generation via Adaptation-Based Programming.
\newblock \emph{IEEE/ACM International Conference on Automated Software Engineering}, pages 380-383, Lawrence, Kansas, November 2011 (short paper, acceptance rate 35\%).

\item
Mohammad Amin Alipour and {\bf Alex Groce}.
\newblock Bounded Model Checking and Feature Omission Diversity.
\newblock \emph{International Workshop on Constraints in Formal Verification}, San Jose, California, November 2011.

\item
Amber Shinsel, Todd Kulesza, Margaret Burnett, William Curran, {\bf Alex Groce}, Simone Stumpf, and Weng-Keen Wong.
\newblock Mini-Crowdsourcing End-User Assessment of Intelligent Assistants: A Cost-Benefit Study.
\newblock \emph{IEEE Symposium on Visual Languages and Human-Centric Computing}, pages 47-54, Pittsburgh, Pennsylvania, September 2011 (acceptance rate 35\%).

\item
Todd Kulesza, Margaret Burnett, Simone Stumpf, Weng-Keen Wong, Shubhomoy Das, {\bf Alex Groce}, Amber Shinsel, Forrest Bice and Kevin McIntosh.
\newblock Where Are My Intelligent Assistant's Mistakes?  A Systematic Testing Approach.
\newblock \emph{International Symposium on End-User Development}, pages 171-186, Brindisi, Italy, June 2011 (acceptance rate 40\%).

\item
{\bf Alex Groce}, Klaus Havelund, and Margaret Smith.
\newblock From Scripts to Specifications: the Evolution of a Flight Software Testing Effort.
\newblock \emph{ACM/IEEE International Conference on Software Engineering}, pages 129-138, Cape Town, South Africa, May 2010 (Software Engineering in Practice, acceptance rate 23\%). 

\item
{\bf Alex Groce}.
\newblock (Quickly) Testing the Tester via Path Coverage.
\newblock \emph{International Workshop on Dynamic Analysis}, Chicago, Illinois, July 2009.

\item
James Andrews, {\bf Alex Groce}, Melissa Weston, and Ru-Gang Xu.
\newblock Random Test Run Length and Effectiveness.
\newblock \emph{IEEE/ACM International Conference on Automated Software Engineering}, pages 19-28, L'Aquila, Italy, September 2008 (acceptance rate 12\%).

\item
Gerard Holzmann, Rajeev Joshi, and {\bf Alex Groce}.
\newblock Tackling Large Verification Problems with the Swarm Tool.
\newblock \emph{SPIN Workshop on Model Checking of Software}, pages 134-143, Los Angeles, California, August 2008.

\item
Klaus Havelund, {\bf Alex Groce}, Gerard Holzmann, Rajeev Joshi, and Margaret Smith.
\newblock Automated Testing of Planning Models.
\newblock \emph{Workshop on Model Checking and Artificial Intelligence}, pages 90-105, Patras, Greece, July 2008.

\item
{\bf Alex Groce} and Rajeev Joshi.
\newblock Random Testing and Model Checking:   Building a Common Framework for Nondeterministic Exploration.
\newblock \emph{International Workshop on Dynamic Analysis}, pages 22-28, Seattle, Washington, July 2008.

\item
{\bf Alex Groce} and Rajeev Joshi.
\newblock Extending Model Checking with Dynamic Analysis.
\newblock \emph{Conference on Verification, Model Checking and Abstract Interpretation}, pages 142-156, San Francisco, California, January 2008 (acceptance rate 34\%).

\item
Nicolas Blanc, {\bf Alex Groce}, and Daniel Kroening.
\newblock Verifying C++ with STL Containers via Predicate Abstraction.
\newblock \emph{IEEE/ACM International Conference on Automated Software Engineering}, pages 521-524, Atlanta, Georgia, November 2007 (short paper, acceptance rate 25\%).

\item
{\bf Alex Groce}, Gerard Holzmann, and Rajeev Joshi.
\newblock Randomized Differential Testing as a Prelude to Formal Verification.
\newblock \emph{ACM/IEEE International Conference on Software Engineering}, pages 621-631, Minneapolis, Minnesota, May 2007 (Software Engineering in Practice, acceptance rate 27\%).

\item
{\bf Alex Groce} and Rajeev Joshi.
\newblock Exploiting Traces in Program Analysis.
\newblock \emph{International Conference on Tools and Algorithms for the Construction and Analysis of Systems}, pages 379-393, Vienna, Austria, March-April 2006 (acceptance rate 25\%), {\bf invited for journal submission to STTT}.

\item
Daniel Kroening, {\bf Alex Groce}, and Edmund Clarke.
\newblock Counterexample Guided Abstraction Refinement via Program Execution.
\newblock \emph{International Conference on
   Formal Engineering Methods},  pages 224-238, Seattle, Washington, November 2004 (acceptance rate 27\%).

\item
Sagar Chaki, {\bf Alex Groce}, and Ofer Strichman.
\newblock Explaining Abstract Counterexamples.
\newblock \emph{ACM SIGSOFT International Symposium on the Foundations of Software Engineering},  pages 73-82, Newport Beach, California, October-November 2004 (acceptance rate 15\%).

\item
{\bf Alex Groce}, Daniel Kroening, and Flavio Lerda.
\newblock Understanding Counterexamples with {\tt explain}.
\newblock \emph{International Conference on Computer Aided Verification},  pages 453-456, Boston, Massachusetts, July 2004 (tool paper).

\item
{\bf Alex Groce} and Daniel Kroening.
\newblock Making the Most of BMC Counterexamples.
\newblock \emph{Workshop on Bounded Model Checking},  pages 71-84, Boston, Massachusetts, July 2004.


\item
{\bf Alex Groce}.
\newblock Error Explanation with Distance Metrics.
\newblock \emph{International Conference on Tools and Algorithms for the Construction and Analysis of Systems}, pages 108-122, Barcelona, Spain, March-April 2004 (acceptance rate 26\%), {\bf invited for journal submission to STTT}.

\item
Sagar Chaki, Edmund Clarke, {\bf Alex Groce}, and Ofer Strichman.
\newblock Predicate Abstraction with Minimum Predicates.
\newblock \emph{Advanced Research Working Conference on Correct Hardware Design and Verification Methods}, pages 19-34, L'Aquila, Italy, October 2003 (acceptance rate 37\%).

\item
Edjard Mota, Edmund Clarke, W. Oliveira, {\bf Alex Groce}, J. Kanda, and M. Falcao.
\newblock VeriAgent: an Approach to Integrating UML and Formal Verification Tools.
\newblock \emph{Brazilian Workshop on Formal Methods}, pages 111-129, Universidade Federal de Campina Grande, Brazil, October 2003.

\item 
Sagar Chaki, Edmund Clarke, {\bf Alex Groce}, Somesh Jha, and Helmut Veith.
\newblock Modular Verification of Software Components in C.
\newblock \emph{ACM/IEEE International Conference on Software Engineering}, pages 385-395, Portland, Oregon, May 2003 (acceptance rate 13\%), {\bf ICSE SIGSOFT Distinguished Paper Award, invited for journal submission to IEEE TSE}. 

\item
{\bf Alex Groce} and Willem Visser.
\newblock What Went Wrong: Explaining Counterexamples.
\newblock \emph{SPIN Workshop on Model Checking of Software}, pages 121-135, Portland, Oregon, May 2003.

\item
{\bf Alex Groce} and Willem Visser.
\newblock Model Checking Java Programs using Structural Heuristics.
\newblock \emph{ACM International Symposium on Software Testing and Analysis}, pages 12-21, Rome, Italy, July 2002 (acceptance rate 19\%).

\item
{\bf Alex Groce}, Doron Peled, and Mihalis Yannakakis.
\newblock AMC: An Adaptive Model Checker.
\newblock \emph{International Conference on Computer Aided Verification}, pages 521-525, Copenhagen, Denmark, July 2002 (tool paper).

\item
{\bf Alex Groce} and Willem Visser.
\newblock Heuristic Model Checking for Java Programs.
\newblock \emph{SPIN Workshop on Model Checking of Software}, pages 242-245, Grenoble, France, April 2002 (tool paper).

\item
{\bf Alex Groce}, Doron Peled, and Mihalis Yannakakis.
\newblock Adaptive Model Checking.
\newblock \emph{International Conference on Tools and Algorithms for the Construction and Analysis of Systems}, pages 357-370, Grenoble, France, April 2002 (acceptance rate 31\%).

\item
Girish Bhat, Rance Cleaveland, and {\bf Alex Groce}.
\newblock Efficient Model Checking Via B\"uchi Tableau Automata.
\newblock \emph{International Conference on Computer Aided Verification}, pages 38-52, Paris, France, July 2001 (acceptance rate 29\%).
\end{Refereed Conference and Workshop Publications}















\begin{Invited Papers}
\item Miroslav Velev, Chaoqiang Zhang, Ping Gao, and {\bf Alex Groce}.
\newblock Exploiting Abstraction, Learning from Random Simulation, and SVM Classification for Efficient Dynamic Prediction of Software Health Problems.
\newblock \emph{International Symposium on Quality Electronic Design}, Santa Clara, CA, March 2015.

\item {\bf Alex Groce}, Alan Fern, Martin Erwig, Jervis Pinto, Tim Bauer, and Mohammad Amin Alipour.
\newblock Learning-Based Test Programming for Programmers.
\newblock \emph{International Symposium On Leveraging Applications of Formal Methods, Verification and Validation}, pages 572-586, Heraclion, Crete, October 2012.


\item Howard Barringer, {\bf Alex Groce}, Klaus Havelund, and Margaret Smith.
\newblock Formal Analysis of Log Files.
\newblock \emph{SMC-IT Workshop on Software Reliability for Space Missions}, Pasadena CA, July 2009.


\item Howard Barringer, {\bf Alex Groce}, Klaus Havelund, and Margaret Smith.
\newblock An Entry Point for Formal Methods:  Specification and Analysis of Event Logs.
\newblock \emph{1st Workshop on Formal Methods in Aerospace}, Electronic Proceedings of Theoretical Computer Science (EPTCS), Eindhoven, Holland, November 2009.

\item Howard Barringer, Klaus Havelund, David Rydeheard, and {\bf Alex Groce}.
\newblock Rule Systems for Runtime Verification: A Short Tutorial.
\newblock \emph{International Workshop on Runtime Verification}, pages 1-24, Grenoble, France, June 2009.

\item Gerard Holzmann, Rajeev Joshi, and {\bf Alex Groce}.
\newblock Swarm Verification.
\newblock \emph{IEEE/ACM International Conference on Automated Software Engineering}, pages 1-6, L'Aquila, Italy, September 2008.

\item {\bf Alex Groce}, Gerard Holzmann, Rajeev Joshi, and Ru-Gang Xu.
\newblock Putting Flight Software Through the Paces with Testing, Model Checking, and Constraint-Solving.
\newblock \emph{International Workshop on Constraints in Formal Verification}, pages 1-15, Sydney, Australia, August 2008.

\item
Gerard Holzmann, Rajeev Joshi, and {\bf Alex Groce}.
\newblock New Challenges in Model Checking.
\newblock \emph
{25 Years of Model Checking}, pages 65-76, Seattle, Washington, August 2006.
\end{Invited Papers}

\begin{Columns, Book Reviews, and Magazine Articles}
\item {\bf Alex Groce}. 
\newblock Simson Garfinkel, Daniel Weise, and Steven Strassman's \emph{The UNIX-Haters Handbook}. 
\newblock Passages column, \emph{ACM SIGSOFT Software Engineering 
  Notes}, accepted for publication.

\item {\bf Alex Groce}. 
\newblock Tom DeMarco and Timothy Lister's \emph{Peopleware: Productive Projects and Teams (Third Edition)}. 
\newblock Passages column, \emph{ACM SIGSOFT Software Engineering 
  Notes}, 43(3): 6-7, October 2018.

\item {\bf Alex Groce}. 
\newblock Confucius' \emph{Analects}. 
\newblock Passages column, \emph{ACM SIGSOFT Software Engineering 
  Notes}, accepted for publication.

\item {\bf Alex Groce}. 
\newblock Bruce Sterling's \emph{Ascendancies: The Best of Bruce Sterling}. 
\newblock Passages column, \emph{ACM SIGSOFT Software Engineering Notes}, 43(2): 6-7, April 2018.

\item {\bf Alex Groce}.
\newblock Karl Popper's \emph{The Logic of Scientific Discovery}.
\newblock Passages column, \emph{ACM SIGSOFT Software Engineering Notes}, 43(1), 4-5, January 2018.

\item {\bf Alex Groce}.
\newblock Brian Kernighan and P. J. Plauger's \emph{The Elements of Programming Style (Second Edition)}.
\newblock Passages column, \emph{ACM SIGSOFT Software Engineering Notes}, 42(4): 5, October 2017.

\item {\bf Alex Groce}.
\newblock Charles Petzold's \emph{Code: The Hidden Language of Computer Hardware and Software}.
\newblock Passages column, \emph{ACM SIGSOFT Software Engineering Notes}, 42(3): 9, July 2017.


\item {\bf Alex Groce}.
\newblock Herbert A. Simon's \emph{The Sciences of the Artificial (Third Edition)}.
\newblock Passages column, \emph{ACM SIGSOFT Software Engineering Notes}, 42(2): 5-6, April 2017.

\item {\bf Alex Groce}.
\newblock Daniel P. Friedman and Matthias Felleisen's \emph{The Little Schemer - 4th edition}.
\newblock Passages column, \emph{ACM SIGSOFT Software Engineering Notes}, 41(6): 5-6, November 2016.

\item {\bf Alex Groce}.
\newblock Jon Bentley's \emph{More Programming Pearls: Confessions of a Coder}.
\newblock Passages column, \emph{ACM SIGSOFT Software Engineering Notes}, 41(5): 6, September 2016.

\item {\bf Alex Groce}.
\newblock Sherry Turkle's \emph{The Second Self: Computers and the Human Spirit}.
\newblock Passages column, \emph{ACM SIGSOFT Software Engineering Notes}, 41(4): 6-7, July 2016

\item {\bf Alex Groce}.
\newblock Samuel C. Florman's \emph{The Existential Pleasures of Engineering}.
\newblock Passages column, \emph{ACM SIGSOFT Software Engineering Notes}, 41(3): 4-5, May 2016.


\item {\bf Alex Groce}.
\newblock Edward R. Tufte's \emph{The Visual Display of Quantitative Information}.
\newblock Passages column, \emph{ACM SIGSOFT Software Engineering Notes}, 41(2): 5, March 2016.

\item {\bf Alex Groce}.
\newblock David Agans' \emph{Debugging: the 9 Indispensable Rules for Finding Even the Most Elusive Software and Hardware Problems}. 
\newblock Passages column, \emph{ACM SIGSOFT Software Engineering Notes}, 41(1): 5, January 2016.


\item {\bf Alex Groce}.
\newblock Donald E. Knuth's \emph{Selected Papers on Computer Science}.
\newblock Passages column, \emph{ACM SIGSOFT Software Engineering Notes}, 40(3): 4-5, May 2015.

\item{\bf Alex Groce}.
\newblock George Polya's \emph{How to Solve It: A New Aspect of Mathematical Method}.
\newblock Passages column, \emph{ACM SIGSOFT Software Engineering Notes}, 40(2): 5-6, March 2015.

\item {\bf Alex Groce}.
\newblock Hugh Kenner's \emph{The Mechanic Muse}.
\newblock Passages column, \emph{ACM SIGSOFT Software Engineering Notes}, 40(1): 8-9, January 2015.

\item {\bf Alex Groce}.
\newblock Andrew Hunt and David Thomas' \emph{The Pragmatic Programmer: from journeyman to master}.
\newblock Passages column, \emph{ACM SIGSOFT Software Engineering Notes}, 39(6): 6-7, November 2014.

\item {\bf Alex Groce}.
\newblock Vernor Vinge's \emph{A Deepness in the Sky}.
\newblock Passages column, \emph{ACM SIGSOFT Software Engineering Notes}, 39(5): 5, September 2014.

\item {\bf Alex Groce}.
\newblock Tom DeMarco and Timothy Lister's \emph{Waltzing with Bears: Managing Risk on Software Projects}.
\newblock Passages column, \emph{ACM SIGSOFT Software Engineering Notes}, 39(4): 8-9, July 2014.

\item {\bf Alex Groce}.
\newblock Henry Petroski's \emph{To Engineer is Human: the Role of Failure in Successful Design}.
\newblock Passages column, \emph{ACM SIGSOFT Software Engineering Notes}, 39(3): 6-7, May 2014.

\item {\bf Alex Groce}.
\newblock Jon Bentley's \emph{Programming Pearls}.
\newblock Passages column, \emph{ACM SIGSOFT Software Engineering Notes}, 39(2): 4-5, March 2014.

\item {\bf Alex Groce}.
\newblock Tracy Kidder's \emph{The Soul of a New Machine}.
\newblock Passages column, \emph{ACM SIGSOFT Software Engineering Notes}, 39(1): 6-7, January 2014.

\item {\bf Alex Groce}.
\newblock Frederick P. Brooks, Jr.'s \emph{The Mythical Man-Month: Essays on Software Engineering}.
\newblock Passages column, \emph{ACM SIGSOFT Software Engineering Notes}, 38(6): 6-7, November 2013.

\item {\bf Alex Groce}.
\newblock Charles Babbage's \emph{Passages from the Life of a Philosopher}.
\newblock Passages column, \emph{ACM SIGSOFT Software Engineering Notes}, 38(5): 17-18, September 2013.
\end{Columns, Book Reviews, and Magazine Articles}

\begin{Technical Reports}
\item
Duc Le, Mohammad Amin Alipour, Rahul Gopinath, and {\bf Alex Groce}.
\newblock Mutation Testing of Functional Programming Languages.
\newblock Technical Report, School of Computer Science and Electrical Engineering, Oregon State University, 2014.

\item
Jamie Andrews, Yihao Ross Zhang, and {\bf Alex Groce}.
\newblock Comparing Automated Unit Testing Strategies.
\newblock Technical Report 736, Department of Computer Science, University of Western Ontario, December 2010.

\item
Brian Kernighan, Dennis Ritchie, Doug McIlroy, Eddie Benowitz, Scott Burleigh, Tim Canham, Benjamin Cichy, Ken Clark, Micah Clark, Len Day, Robert Denise, Will Duquette, Dan Dvorak, Dan Eldred, Ed Gamble, Peter Gluck, Kim Gostelow, Chris Grasso, {\bf Alex Groce}, Dave Hecox, Gerard Holzmann, Joe Hutcherson, Rajeev Joshi, Roger Klemm, Frank Kuykendall, Mary Lam, Steve Larson, Todd Litwin, Tom Lockhart, Lloyd Manglapus, Kenny Meyer, Alex Murray, Al Niessner, Bob Rasmussen, Len Reder, Glenn Reeves, Kirk Reinholtz, Mike Roche, Nicolas Rouquette, Steve Scandore, Marcel Schoppers, Dave Smyth, Ken Starr, Igor Uchenik, Dave Wagner, Garth Watney, Steve Watson, Matt Wette, and Jesse Wright.
\newblock JPL Institutional Coding Standard for the C Programming Language.
\newblock Jet Propulsion Laboratory, web publication, March 3, 2009 (\url{http://lars-lab.jpl.nasa.gov/JPL\_Coding\_Standard\_C.pdf}).

\item
Nicolas Blanc, Daniel Kroening, and {\bf Alex Groce}.
\newblock Verifying C++ with STL Containers via Predicate Abstraction.
\newblock Technical Report 506, ETH Z\"urich, January 2006.

\item
{\bf Alex Groce}.
\newblock Error Explanation and Fault Localization with Distance Metrics.
\newblock (Ph.D. Thesis) Technical Report CMU-CS-05-121, Carnegie Mellon University, March 2005.

\item
{\bf Alex Groce}, Doron Peled, and Mihalis Yannakakis.
\newblock {AMC:} An Adaptive Model Checker.
\newblock ALR-2002-008, Avaya Labs Research, February 2002.

\item
{\bf Alex Groce} and Willem Visser.
\newblock What Went Wrong: Explaining Counterexamples.
\newblock Technical Report 02-08, RIACS, USRA, February 2002.

\item
{\bf Alex Groce}, Doron Peled, and Mihalis Yannakakis.
\newblock Adaptive Model Checking.
\newblock ALR-2002-002, Avaya Labs Research, January 2002.

\item
Sergey Berezin and {\bf Alex Groce}.
\newblock SyMP: The Hacker's Manual.
\newblock Carnegie Mellon University, web publication, May 12, 2001 (\url{http://www.cs.cmu.edu/~modelcheck/symp.html}).

\item
Sergey Berezin and {\bf Alex Groce}.
\newblock SyMP: The User's Guide.
\newblock Carnegie Mellon University, web publication, December 27, 2000 (\url{http://www.cs.cmu.edu/~modelcheck/symp.html}).
\end{Technical Reports}

\begin{Professional Activities and Service}
  \item ACM TOSEM (Transactions on Software Engineering and
    Methodology) Board of Distinguished Reviewers, February 2019-present.
  
  \item Associate Editor for Software Testing, IEEE Software,
    September 2018-present.

  \item Reviewer for \emph{IEEE Transactions on Software Engineering} (TSE), \emph{ACM
   Transactions
   on  Software  Engineering  and  Methodology} (TOSEM), \emph{Journal of the ACM} (JACM), 
  \emph{Software Tools for Technology Transfer} (STTT), \emph{Formal
    Methods in System Design} (FMSD), \emph{IEEE Transactions on
    Parallel and Distributed Systems} (TPDS), \emph{IEEE Transactions
    on Reliability}, \emph{IEEE Transactions on Embedded Computing
    Systems} (TECS), \emph{Information Processing Letters} (IPL),
  \emph{IEEE Transactions on Computers} (TC), \emph{IEEE Transactions
    on Industrial Informatics}, \emph{Empirical Software Engineering},
  \emph{Acta Informatica}, \emph{ACM Computing Surveys},
  \emph{Algorithmica}, \emph{Computers \& Security},
  \emph{Microprocessors and Microsystems: Embedded Hardware Design},
  \emph{Journal on Satisfiability, Boolean Modeling and Computation}
  (JSAT), \emph{Journal of Logic and Computation}, \emph{Artificial
    Intelligence}, \emph{Journal of Computer and System Sciences}
  (JCSS), \emph{Software Testing, Verification and Reliability}
  (STVR), \emph{Automated Software Engineering Journal}, \emph{Annals
    of Mathematics and Artificial Intelligence} (AMAI), \emph{Science
    of Computer Programming}, \emph{ACM Computing Surveys}, \emph{Journal of Applied Logic} (JAL),
  \emph{Journal of Computer Science and Technology} (JCST),
  \emph{Journal of Systems and Software} (JSS), \emph{PLOS ONE},
  \emph{Information and Software Technology} (IST), \emph{The Computer
    Science Journal}, \emph{Journal of Software: Evolution and Process},\emph{International Conference on Computer Aided
  Verification} (CAV), \emph{International Conference on Tools and Algorithms for the Construction and Analysis of Systems} (TACAS), \emph{ACM SIGSOFT Symposium on the Foundations of Software Engineering} (FSE), \emph{IEEE/ACM International Conference on Automated Software Engineering} (ASE), \emph{ACM/IEEE International Conference on Software Engineering} (ICSE), \emph{ACM SIGPLAN - SIGACT Symposium on Principles of Programming Languages} (POPL), \emph{ACM SIGPLAN Conference on Programming Language Design and Implementation} (PLDI), \emph{ACM CHI Conference on Human Factors in Computing Systems} (CHI), \emph{Verification, Model Checking, and Abstract Interpretation} (VMCAI), \emph{International Conference on Software Testing, Verification and Validation} (ICST), \emph{Symposium on Automated and Analysis-driven Debugging} (AADEBUG), \emph{Fundamental Approaches to Software Engineering} (FASE), \emph{NASA Formal Methods Symposium} (NFM), \emph{ACM SIGCHI Symposium on Engineering Interactive Computing Systems} (EICS), \emph{Formal Techniques for Networked and Distributed Systems} (FORTE), \emph{Asia and South Pacific Design Automation Conference} (ASP-DAC), \emph{Logic for Programming, Artificial Intelligence, and Reasoning} (LPAR), \emph{Formal Methods in Computer-Aided Design} (FMCAD), \emph{Australasian Computer Science Conference} (ACSC), \emph{SPIN Workshop on Model Checking of Software}, \emph{Workshop on Model Checking and Artificial Intelligence} (MoChArt), \emph{Specification and Verification of Component-Based Systems} (SAVCBS), \emph{Workshop on Verification and Debugging} (V\&D), \emph{International Workshop on Constraints in Formal Verification} (CFV), \emph{Workshop on Software Model Checking} (SoftMC), \emph{Workshop on Verification and Validation for Planning and Scheduling Systems} (VVPS), \emph{IEEE Software Engineering Workshop} (SEW), \emph{Java Pathfinder Workshop} (JPF), \emph{International Workshop on Dynamic Analysis} (WODA), and \emph{Workshop on Formal Methods for Industrial Critical Systems} (FMICS).
\item Senior member of ACM (Association for Computing Machinery).
  \item ACM, Special Interest Group on Software Engineering (SIGSOFT),
    Special Interest Group on Programming Languages (SIGPLAN), Special
    Interest Group on Embedded Systems (SIGBED) member.
\item ACM SIGSOFT ``Passages'' classic book review columnist for SIGSOFT Software Engineering Notes.
\item IEEE, IEEE Computer Society member.
\item External Reviewer for Natural Sciences and Engineering Research Council of Canada.
\item Reviewer for NASA Small Business Innovation Research (SBIR) program.
\item External Reviewer for Israel Science Foundation.
\item External Reviewer for South African National Research Foundation.
\end{Professional Activities and Service}

\begin{Panel and Committee Service}
\item Program committee $\cdot$ 13th IEEE International Conference on Software Testing, Validation, and Verification (ICST'20), Porto, Portugal, March 2020.
\item Workshops co-chair $\cdot$ ACM SIGSOFT International Symposium on Software Testing and Analysis (ISSTA'19), Beijing, China, July 2019.
\item Program committee $\cdot$ 41st ACM/IEEE International Conference on Software Engineering (ICSE'19), Montreal, Canada, May 2019. 
\item Program committee $\cdot$ 33rd IEEE/ACM International Conference on Automated Software Engineering (ASE'18), Montpellier, France, September 2018.
\item Program committee $\cdot$ 40th ACM/IEEE International Conference on Software Engineering (ICSE'18), Gothenburg, Sweden, May 2018. 
\item Program committee $\cdot$ 17th International Conference on Runtime Verification (RV'17), Seattle, Washington, September 2017.
\item Program committee $\cdot$ ACM SIGSOFT International Symposium on Software Testing and Analysis (ISSTA'17), Santa Barbara, California, July 2017.
\item Program committee $\cdot$ Doctoral Symposium, ACM SIGSOFT International Symposium on Software Testing and Analysis (ISSTA'17), Santa Barbara, California, July 2017.
\item Program committee $\cdot$ 2017 Summer Computer Simulation Conference (SCSC'17), Bellevue, Washington, July 2017.
\item Co-chair $\cdot$  2nd Workshop on Causal Reasoning for Embedded
  and safety-critical Systems Technologies (CREST'17), Uppsala,
  Sweden, April 2017; co-located with European Joint Conferences on
  Theory \& Practice of Software (ETAPS).
\item Steering committee $\cdot$ International Workshop on Dynamic Analysis, 2013-2016.
\item Program committee $\cdot$ 23rd International SPIN Symposium on Model Checking of Software (SPIN'16), Eindhoven, the Netherlands, April 2016.
\item Program chair $\cdot$ 9th International Workshop on Constraints in Formal Verification, Austin, Texas, November 2015; co-located with IEEE/ACM International Conference on Computer-Aided Design.
\item Program committee $\cdot$ ACM SIGDA Student Research Competition, 34th IEEE/ACM International Conference on Computer-Aided Design, November 2015.
%\item Program committee $\cdot$ 3rd IEEE International Workshop on Formal Methods Integration, San Francisco, California, August 2015.
\item Program committee $\cdot$ 22nd International SPIN Symposium on Model Checking of Software (SPIN'15), Stellenbosch, South Africa, August 2015.
\item Chair $\cdot$ Doctoral symposium at ACM SIGSOFT International Symposium on Software Testing and Analysis (ISSTA'15), Baltimore, Maryland, July 2015. 
%\item Program committee, $\cdot$ 7th Working Conference on Verified Software:  Theories, Tools, and Experiments (VSTTE'15), San Francisco, California, July 2015.
\item Workshops selection committee $\cdot$ 37th ACM/IEEE International Conference on Software Engineering (ICSE'15), Firenze, Italy, May 2015.
\item Reviewing committee $\cdot$ 37th ACM/IEEE International Conference on Software Engineering (ICSE'15), Firenze, Italy, May 2015.
\item Program committee $\cdot$ 7th NASA Formal Methods Symposium (NFM'15), Pasadena, California, April 2015. 
\item Publicity committee $\cdot$ 8th IEEE International Conference on Software Testing, Verification and Validation (ICST'15), Graz, Austria, April 2015.
%\item Program committee $\cdot$ 5th IEEE Workshop on Program Debugging (IWPD'14), Naples, Italy, November 2014.
%\item Program committee $\cdot$ Java Pathfinder Workshop 2014 (JPF'14), Salt Lake City, Utah, November 2014.
\item Program committee $\cdot$ Research Tool Demonstrations for 22nd ACM
  SIGSOFT International Symposium on the Foundations of Software
  Engineering (FSE-DEMO'14), Hong Kong, November 2014.
\item Program committee $\cdot$ ACM SIGDA Student Research Competition, 33rd IEEE/ACM International Conference on Computer-Aided Design, November 2014.
%\item Program committee $\cdot$ 36th Annual IEEE Software Engineering Workshop (SEW-36), Mountain View, California, August 2014.
\item Program committee $\cdot$ 21st International SPIN Symposium on Model Checking of Software (SPIN'14), San Jose, California, July 2014.
\item Program committee $\cdot$ ACM SIGSOFT International Symposium on Software Testing and Analysis (ISSTA'14), San Jose, California, July 2014.
\item Program committee $\cdot$ 28th IEEE/ACM International Conference on Automated Software Engineering (ASE'13), Palo Alto, California, November 2013.
%\item Program committee $\cdot$ 8th International Workshop on Constraints in Formal Verification (CFV'13), San Jose, California, November 2013.
\item Co-chair $\cdot$ 11th International Workshop on Dynamic Analysis (WODA'13), Houston, Texas, March 2013; co-located with ACM International Conference on Architectural Support for Programming Languages and Operating Systems (ASPLOS).
\item Workshops selection committee $\cdot$ 35th ACM/IEEE International Conference on Software Engineering (ICSE'13), San Francisco, California, May 2013.
%\item Program committee $\cdot$ Java Pathfinder Workshop 2012 (JPF'12), Raleigh, North Carolina, November 2012.
%\item Program committee $\cdot$ 35th Annual IEEE Software Engineering Workshop (SEW-35), Heraclion, Crete, October 2012.
%\item Program committee $\cdot$ 10th International Workshop on Dynamic Analysis (WODA'12), Minneapolis, Minnesota, July 2012.
%\item Program committee $\cdot$ 19th International Workshop on Model Checking Software (SPIN'12), Oxford, England, July 2012.
%\item Program committee $\cdot$ 7th International Workshop on Constraints in Formal Verification (CFV'11), San Jose, California, November 2011.
%\item Program committee $\cdot$ Java Pathfinder Workshop 2011 (JPF'11), Lawrence, Kansas, November 2011.
\item Co-chair $\cdot$ 18th International Workshop on Model Checking Software (SPIN'11), Snowbird, Utah, July 2011; co-located with International Conference on Computer Aided Verification (CAV).
%\item Program committee $\cdot$ 3rd Workshop on Verification and Validation for Planning and Scheduling Systems (VVPS'11), Freiburg, Germany, June 2011.
%\item Program committee $\cdot$ 34th Annual IEEE Software Engineering Workshop (SEW-34), Limerick, Ireland, June 2011.
\item Program committee $\cdot$ 3rd NASA Formal Methods Symposium (NFM'11), Pasadena, California, April 2011. 
\item Program committee $\cdot$ 14th International Conference on Fundamental Approaches to Software Engineering (FASE'11), Saarbrucken, Germany, March 2011.
%\item Program committee $\cdot$ 9th Workshop on Specification and Verification of Component-Based Systems (SAVCBS'10), Santa Fe, New Mexico, November 2010.
%\item Program committee $\cdot$ 8th Workshop on Specification and Verification of Component-Based Systems (SAVCBS'09), Amsterdam, the Netherlands, August 2009.
%\item Program committee $\cdot$ 6th International Workshop on Constraints in Formal Verification (CFV'09), Grenoble, France, June 2009.
%\item Program committee $\cdot$ 7th Workshop on Specification and Verification of Component-Based Systems (SAVCBS'08), Atlanta, Georgia, November 2008.
%\item Program committee $\cdot$ 14th International Workshop on Model Checking Software (SPIN'07), Berlin, Germany, July 2007.
%\item Program committee $\cdot$ 1st International Workshop on Verification and Debugging (V\&D'06), Seattle, Washington, August 2006.
\item Program committee $\cdot$ 7th IEEE International Workshop on
  Formal Methods Integration (FMi'19), Los Angeles, California,
  July-August 2019; Second ACM International Workshop on
  the Engineering of Reliable, Robust, and Secure Embedded Wireless
  Sensing Systems (FAILSAFE'18), Shenzen, China, November 2018; IEEE
  International Workshop on Debugging and Repair (IDEAR'18), Memphis,
  Tennessee, October 2018; Workshop on Testing, Analysis, and Verification of Cyber-Physical Systems and Internet of Things (TAV-CPS/IoT'18), Amsterdam, the Netherlands, July 2018; 6th IEEE International Workshop on Formal Methods Integration (FMI'18), Salt Lake City, Utah, July 2018;  International Workshop on Software Fairness, (FairWare'18), Gothenburg, Sweden, May 2018; 3rd Workshop on formal reasoning about Causation, Responsibility, and Explanations in Science and Technology (CREST'18), Thessaloniki, Greece, April 2018; 8th IEEE International Workshop on Program Debugging (IWPD'17), Toulouse, France, October 2017; 5th IEEE International Workshop on Formal Methods Integration, San Diego, California, August 2017; 7th IEEE International Workshop on Program Debugging (IWPD'16), Ottawa, Canada, October 2016; 4th IEEE International Workshop on Formal Methods Integration, Pittsburgh, Pennsylvania, April 2016; 1st Workshop on Causal-based Reasoning for Embedded and Safety-critical Systems Technologies (CREST'16), Eindhoven, The Netherlands, April 2016; Java Pathfinder Workshop 2015 (JPF'15); 6th IEEE Workshop on Program Debugging (IWPD'15), Gaithersburg, Maryland, November 2015; 3rd IEEE International Workshop on Formal Methods Integration, San Francisco, California, August 2015; 7th Working Conference on Verified Software:  Theories, Tools, and Experiments (VSTTE'15), San Francisco, California, July 2015; 5th IEEE Workshop on Program Debugging (IWPD'14), Naples, Italy, November 2014; Java Pathfinder Workshop 2014 (JPF'14), Salt Lake City, Utah, November 2014; 36th Annual IEEE Software Engineering Workshop (SEW-36), Mountain View, California, August 2014; 8th International Workshop on Constraints in Formal Verification (CFV'13), San Jose, California, November 2013; Java Pathfinder Workshop 2012 (JPF'12), Raleigh, North Carolina, November 2012; 35th Annual IEEE Software Engineering Workshop (SEW-35), Heraclion, Crete, October 2012; 10th International Workshop on Dynamic Analysis (WODA'12), Minneapolis, Minnesota, July 2012; 19th International Workshop on Model Checking Software (SPIN'12), Oxford, England, July 2012; 7th International Workshop on Constraints in Formal Verification (CFV'11), San Jose, California, November 2011; Java Pathfinder Workshop 2011 (JPF'11), Lawrence, Kansas, November 2011; 3rd Workshop on Verification and Validation for Planning and Scheduling Systems (VVPS'11), Freiburg, Germany, June 2011; 34th Annual IEEE Software Engineering Workshop (SEW-34), Limerick, Ireland, June 2011; 9th Workshop on Specification and Verification of Component-Based Systems (SAVCBS'10), Santa Fe, New Mexico, November 2010; 8th Workshop on Specification and Verification of Component-Based Systems (SAVCBS'09), Amsterdam, the Netherlands, August 2009; 6th International Workshop on Constraints in Formal Verification (CFV'09), Grenoble, France, June 2009; 7th Workshop on Specification and Verification of Component-Based Systems (SAVCBS'08), Atlanta, Georgia, November 2008; 14th International Workshop on Model Checking Software (SPIN'07), Berlin, Germany, July 2007; 1st International Workshop on Verification and Debugging (V\&D'06), Seattle, Washington, August 2006.
\end{Panel and Committee Service}


\begin{Teaching}
\item {\bf Summary:} I developed and taught classes covering testing, analysis, and software engineering to graduate and undergraduate students.  My teaching evaluations over these classes and during my six semesters as a teaching assistant have been strongly positive.  I served on the undergraduate curriculum committee at Oregon State from 2009 until 2015, and (co-)chaired the committee from September 2015-2016.

\item{\bf Spring 2018} $\cdot$ Associate Professor, Northern Arizona
  University.  Taught CS 499, a Special Topics course on software
  security, covering basic security concepts, protocols and protocol
  vulnerabilities, famous code vulnerabilities (Heartbleed and goto fail), AFL and other
  fuzzing tools, taint analysis, static and dynamic approaches to
  detecting vulnerabilities, and the vulnerability-to-exploit path. 

\item{\bf Spring 2017} $\cdot$ Associate Professor, Northern Arizona University.  Taught CS 499, a Special Topics course in automated software test generation.  Topics included basics of automated software testing, and using actual tools in the field, including TSTL, Csmith, afl-fuzz, and Hypothesis.  Class included guest lectures by Linux kernel developers, testing tool creators, and security-based testing experts.

\item{\bf Fall 2016} $\cdot$ Associate Professor, Oregon State University.  Taught CS 361, first in software engineering sequence, with emphasis on software architecture, testability, foundations of software engineering as a discipline (readings including Brooks, Parnas, Butler Lampson, and DeMarco and Lister).

\item{\bf Fall, Winter 2013} $\cdot$ Assistant Professor, Oregon State University.  Developed ecampus online version of class in Applied Software Engineering, focusing on testing, analysis, code review, debugging, and software maintenance; recorded presentations and selected materials for online learning.

\item{\bf Winter, Spring 2010-2015} $\cdot$  Assistant Professor, Oregon State University.  Taught CS 362 and CS 562, undergraduate and graduate classes in Applied Software Engineering.  Focused on theory and practice of software implementation, including maintenance, code management, static analysis, testing, model checking, hardware interface and simulation, and debugging.  Project-centered courses featured use of an open-source social-networking/project repository system and  innovative exchange of programs for testing.  Developed upper-level class on software security (basic concepts, protocols, security exploits) and reliability.  Developed and recorded online version of class on software testing, analysis, and verification.  Taught graduate seminar on Static Analysis and Model Checking (CS 569), with focus on software security and combining static and dynamic analysis.

\item{\bf Fall 2009} $\cdot$ Assistant Professor, Oregon State
  University.  Developed required courses (to be taught Winter and
  Spring terms) on applied software engineering for undergraduate and
  graduate students (CS 362 and CS 562), with a focus on design for
  testability, practical debugging and maintenance, test-driven
  development, code analysis and instrumentation, and automated
  testing and verification. Covered test-driven development as guest
  lecturer in first undergraduate software engineering course (CS
  361). Mentored graduate students and initiated a research program
  involving graduate and undergraduate students.  Served on
  undergraduate curriculum committee. Helped develop new undergraduate
  concentration in software engineering for sustainability and energy
  management.

  \item {\bf Spring 2008, Spring 2009} $\cdot$ Lecturer, California
  Institute of Technology, \emph{CS 119 Reliable Software: Testing and
  Monitoring} (developed and taught with Klaus Havelund), third term
  2007-2008 and 2008-2009.  Topics included random testing,
  constraint-based testing, coverage measures, design for testability, static analysis,
  test-driven development, automated debugging, and the use of model
  checkers.  Focused on practical application (and limits)
  of state-of-the-art methods.

  \item  {\bf Spring 2007} $\cdot$ External Master's thesis examiner, Stellenbosch University

   \item {\bf Fall 2003} $\cdot$ Teaching assistant, Carnegie Mellon
   University, for undergraduate course 15-212, \emph{Principles of
   Programming} (introduction to programming in Standard ML, including type
   discipline and proof by induction): Formulated assignments, graded
   assignments and tests, taught a weekly recitation session, and held
   office hours.

   \item {\bf Spring 2000} $\cdot$ Teaching assistant, Carnegie Mellon
   University, for undergraduate course 15-312, \emph{Foundations of
   Programming Languages} (advanced type theory, continuations, and concurrency): Formulated tests and assignments, graded
   assignments and tests, lectured in absence of Professor Harper,
   taught a weekly recitation session, and held office hours.

   \item {\bf Fall 1998} $\cdot$ Teaching assistant, North Carolina
   State University, for undergraduate course CSC417, \emph{Theory of
   Programming Languages} (type theory and functional programming in ML):
   Graded assignments and held weekly office hours.

   \item {\bf Fall 1997, Spring 1998} $\cdot$ Teaching assistant,
   North Carolina State University, for undergraduate course CSC210,
   {\em Programming Concepts} (second-level introductory course in
   C++, including pointers, recursion, and fundamental data
   structures): Graded assignments, held weekly office hours, and
   provided on-the-spot teaching and assistance to students in
   computer labs.

   \item {\bf Spring 1997} $\cdot$ Teaching assistant, North
   Carolina State University, CSC495C, {\em
   Special Topics} (class for professional C programmers learning C++ and object-oriented design):  Graded assignments and held weekly office
   hours.

\end{Teaching}

\begin{Current Students}
\item Arpit Christi, Oregon State University, PhD, Committee Chair 
%\item He Xhang, Oregon State University, PhD, Committee Chair 
%\item Qi Qi, Oregon State University, MS, Major Advisor
%\item Gokul Caushik, Oregon State University, MS, Major Advisor
%\item Sheng Chen, Oregon State University, PhD, Committee Member
\item Alex Wiggins, Oregon State University, PhD, Committee Member
\item Austin Sanders, Northern Arizona University, PhD, Committee Member
\end{Current Students}

\begin{Graduated Students}
\item Rahul Gopinath, Oregon State University, PhD, Committee Co-Chair
\item Mohammad Amin Alipour, Oregon State University, PhD, Committee Chair 
\item Ali Aburas, Oregon State University, PhD, Committee Chair
\item Chaoqiang Zhang, Oregon State University, PhD, Committee Chair
\item Xin Liu, Oregon State University, MS, Committee Chair 
\item Kazuki Kaneoka, Oregon State University, MS, Committee Chair 
\item Shalini Shamasunder, Oregon State University, MS, Committee Chair
\item Gokul Caushik, Oregon State University, MEng, Major Advisor
\item Aravind Palanisami, Oregon State University, MEng, Major Advisor
\item Pengfei Chen, Oregon State University, MEng, Major Advisor
\item Iftekhar Ahmed, Oregon State University, PhD, Committee Member
\item Soroush Ghorashi, Oregon State University, PhD, Committee Member
\item Jervis Pinto, Oregon State University, PhD, Committee Member
\item Todd Kulesza, Oregon State University, PhD, Committee Member
\item Chris Chambers, Oregon State University, PhD, Committee Member
\item Yang Chen, University of Utah, PhD, External Committee Member
\item Christopher Bogart, Oregon State University, PhD, Committee Member
\item Eric Walkingshaw, Oregon State University, PhD, Committee Member
\item Duc Le, Oregon State University, MS, Committee Member
\item Darren Forrest, Oregon State University, MS, Committee Member
\item Prashanth Ayyavu, Oregon State University, MS, Committee Member
\item Nishanthini Narayanan, Oregon State University, MS, Committee Member
\item Nitin Mohan, Oregon State University, MS, Committee Member
\item David Burri, Oregon State University, MS, Committee Member
\item Madhura Vadvalkar, Oregon State University, MEng, Committee Member
\item Michael Tichenor, Oregon State University, MS, Committee Member
\item Alex Diede, Oregon State University, MEng, Committee Member
\end{Graduated Students}



\begin{Invited Seminars}
\item Invited speaker, 7th Halmstad Summer School on Testing, Halmstad, Sweden, June 12-15, 2017.

  \item Dagstuhl Seminar 03491, Understanding Program Dynamics, Schloss Dagstuhl, Wadern, Germany, November 31-December 5, 2003.
\end{Invited Seminars}

\begin{Invited Talks and Panels}
\item ``TSTL: a Little (Integrated) Language for Testing'', 7th Halmstad Summer School on Testing, Halmstad, Sweden, June 12, 2017.

\item ``Presenting TSTL (and the Quest for One Test to Rule them All),'' School of Informatics, Computing, and Cybersystems, Northern Arizona University, Flagstaff, AZ, January 22, 2016.

\item ``Understanding and Exploiting Triggers and Suppressors in Testing,'' Galois, Inc., Portland, OR, March 3, 2014.

\item ``Making the Most of Random Tests,'' Google Inc., Mountain View, CA, February 24, 2014.

\item ``Effective Random Testing for Critical Systems Software,'' Arizona State University, Phoenix, AZ, January 13, 2014.

\item Panelist, ``Program Debugging: Transitioning from Research to Practice,'' International Workshop on Program Debugging, Pasadena, CA, November 4, 2013. 

\item ``Learning-Based Test Programming for Programmers,'' International Symposium On Leveraging Applications of Formal Methods, Verification and Validation, Heraclion, Crete, October 16, 2012.

\item ``For Truly Thorough Testing, You Have to Leave Things Out,'' Northern Arizona University, Flagstaff, AZ, April 11, 2012.

\item ``Traces in Spaces:   You Can Learn a Lot About a Program by Running It,'' School of Electrical Engineering and Computer Science Colloquium Series, Oregon State University, Corvallis, OR, February 2, 2009.

\item ``Putting Flight Software Through the Paces with Testing, Model Checking, and Constraint-Solving,'' International Workshop on Constraints in Formal Verification / International Verification Workshop, Sydney, Australia, August 11, 2008.

\item ``Asking the Right Questions --- and Understanding the Answers --- in Software Testing,'' (with Klaus Havelund), Information Science and Technology Lunch Bunch, California Institute of Technology, Pasadena, CA, February 19, 2008.

\item ``How to Break a (Flash) File System,'' Jet Propulsion Laboratory-Goddard Space Flight Center (JPL-GSFC) Quality Mission Software Workshop, Santa Barbara, CA, May 2, 2006.

 \item ``Exploiting Traces in Program Analysis,'' Workshop on Theories, Methods and Tools for Building Systems from Interacting Components, California Institute of Technology, Pasadena, CA, October 31, 2005.


  \item ``Explaining Counterexamples,'' IBM T. J. Watson Research Center, Hawthorne, NY, December 20, 2004.

  \item ``Explaining Counterexamples,'' Microsoft Research, Redmond, WA, November 8, 2004.  Similar version presented as Speakers' Club seminar at Carnegie Mellon University, Pittsburgh, PA, December 9, 2004.

  \item ``Debugging Code with Model Checkers,'' Jet Propulsion Laboratory, Pasadena, CA, November 1, 2004.

\item ``Error Explanation via Model Checking,''  Dagstuhl Seminar 03491, Understanding Program Dynamics, Schloss Dagstuhl, Wadern, Germany, December 5, 2003.



\end{Invited Talks and Panels}

\begin{Selected Presentations}
\item ``Towards Automated Composition of Heterogeneous Tests for Cyber-Physical Systems,'' Workshop on Testing Embedded and Cyber-Physical Systems, Santa Barbara, CA, July 13, 2017.

\item ``A Suite of Tools for Making Effective Use of Automatically Generated Tests,'' International Symposium on Software Testing and Analysis, Santa Barbara, CA, July 10, 2017.

\item ``One Test to Rule Them All,'' International Symposium on Software Testing and Analysis, Santa Barbara, CA, July 10, 2017.

\item ``Mitigating (and Exploiting) Test Reduction Slippage,'' Workshop on Automated Software Testing, Seattle, WA, November 18, 2016.

\item ``How Verified is My Code?  Falsification-Driven Verification,'' IEEE/ACM International Conference on Automated Software Engineering, Lincoln, NE, November 13, 2015.

\item ``A Little Language for Testing,'' NASA Formal Methods Symposium, Pasadena, CA, April 28, 2015.

\item ``Coverage and Its Discontents,'' Onward! Essays, part of SPLASH (ACM SIGPLAN Conference on Systems, Programming, Languages and Applications: Software for Humanity), Portland, OR, October 24, 2014.

\item ``Help! Help! I'm Being Suppressed! The Significance of Suppressors in Software Testing,'' IEEE International Symposium on Software Reliability Engineering, Pasadena, CA, November 6, 2013.

\item ``New Directions in Random Testing: From Mars Rovers to JavaScript Engines,'' Galois, Inc. Tech Talk, Portland, OR, September 12, 2013.

\item ``Beyond the Kitchen Sink: Swarm Testing'', Jet Propulsion Laboratory, Pasadena, CA, May 7, 2013.

\item ``Extended Program Invariants: Applications in Testing and Fault Localization,'' International Workshop on Dynamic Analysis, Minneapolis, MN, July 15, 2012.

\item ``Finding Common Ground: Choose, Assert, and Assume,'' International Workshop on Dynamic Analysis, Minneapolis, MN, July 15, 2012.

\item ``Coverage Rewarded: Test Input Generation via Adaptation-Based Programming,'' IEEE/ACM International Conference on Automated Software Engineering, Lawrence, KS, November 9, 2011.

\item ``Establishing Appropriate User Trust in Machine-Learned Classifiers,'' Human/Machine Learning Partnerships, Oregon State University, Corvallis, OR, May 21, 2010.

\item ``Can End Users Test Machine-Learning Classifiers?,'' End Users and Machine Learning Day, Oregon State University, Corvallis, OR, February 26, 2010.

\item ``Path Coverage and Its Discontents,'' School of Electrical Engineering and Computer Science Colloquium Series, Oregon State University, Corvallis, OR, February 22, 2010.

\item ``(Quickly) Testing the Tester via Path Coverage,'' International Workshop on Dynamic Analysis, Chicago, IL, July 20, 2009.

\item ``Advanced Testing Tools,'' (with Klaus Havelund), Engineering and Science Directorate - Software Engineering Process Group, Jet Propulsion Laboratory, Pasadena CA, April 30, 2009.

\item ``Random Testing and Model Checking:   Building a Common Framework for Nondeterministic Exploration,'' International Workshop on Dynamic Analysis, Seattle, WA, July 21, 2008.

\item ``Model-Driven Software Verification Methods,'' LaRS Advisory Committee Meeting, Jet Propulsion Laboratory, Pasadena, CA, June 26, 2008.


\item ``Model Checking, Dynamic Analysis, and Unsound Abstractions,'' Southern California Workshop on Programming Languages and Systems, Claremont, CA, February 2, 2008.

\item ``Extending Model Checking with Dynamic Analysis,'' Verification, Model Checking and Abstract Interpretation, San Francisco, CA, January 8, 2008.

\item ``Model-Driven Verification,'' Mission Computing and Autonomy Systems Research Program (982) FY07 Year End Review, Jet Propulsion Laboratory, Pasadena, CA, October 3, 2007.

\item ``Testing the Kepler Flash File System,'' LaRS Advisory Committee Meeting, Jet Propulsion Laboratory, Pasadena, CA, July 27, 2007.

\item ``Randomized Differential Testing as a Prelude to Formal Verification,'' ACM/IEEE International Conference on Software Engineering, Minneapolis, MN, May 24, 2007.

\item ``Strengthening Software Testing,'' LaRS Advisory Committee Meeting, Jet Propulsion Laboratory, Pasadena, CA, July 26, 2006.  Similar version presented as Section 316 Brown Bag Lecture at JPL on August 23, 2006.

\item ``LaRS File System Test Approach,'' Flight Software Applications and Data Management (316D) Group Meeting, Jet Propulsion Laboratory, Pasadena, CA, May 16, 2006.



 \item ``Exploiting Traces in Program Analysis,'' International Conference on Tools and Algorithms for the Construction and Analysis of Systems, Vienna, Austria, March 29, 2006.


 \item ``Bounded Model Checking Explained,'' LaRS Seminar, Jet Propulsion Laboratory, Pasadena, CA, June 14, 2005.

 \item ``Error Explanation and Fault Localization with Distance Metrics,'' Thesis Oral, Carnegie Mellon University, Pittsburgh, PA, March 3, 2005.



  \item ``Counterexample Guided Abstraction Refinement via Program Execution,'' International Conference on Formal Engineering Methods, Seattle, WA, November 11, 2004.



  \item ``Explaining Abstract Counterexamples,''ACM SIGSOFT International Symposium on the Foundations of Software Engineering, Newport Beach, CA, November 2, 2004.



  \item ``CBMC and C Model Checking,'' MURI (Multidisciplinary University Research Initiative) Review Meeting, Annapolis, MD, August 16, 2004.

  \item ``Java PathFinder,'' Software Model Checking Seminar, Carnegie Mellon University, Pittsburgh, PA, July 22, 2004.

  \item ``Error Explanation with Distance Metrics,'' International Conference on Tools and Algorithms for the Construction and Analysis of Systems, Barcelona, Spain, March 29, 2004.  



  \item ``Explaining Errors,'' MURI (Multidisciplinary University Research Initiative) Workshop, Carnegie Mellon University, Pittsburgh, PA, July 22, 2003.

  \item ``Explaining Counterexamples: Causal Analysis and Comparison of Transition Sequences,''  Specification and Verification Center, Carnegie Mellon University, Pittsburgh, PA, May 20, 2003.

  \item ``What Went Wrong: Explaining Counterexamples,''  SPIN Workshop on Model Checking of Software, Portland, OR, May 9, 2003.  Earlier versions presented at Specification and Verification Center, Carnegie Mellon University, Pittsburgh, PA, September 17, 2002, and NASA Ames Research Center/RIACS Seminar, Mountain View, CA, August 8, 2002.

%  \item ``What Went Wrong?''  Specification and Verification Center, Carnegie Mellon University, Pittsburgh, PA, September 17, 2002.

%\item ''What Went Wrong?''  NASA Ames Research Center/RIACS Seminar, Mountain View, CA, August 8, 2002.

  \item ``Model Checking Java Programs using Structural Heuristics,''  International Symposium on Software Testing and Analysis, Rome, Italy, July 22, 2002.

  \item ``Heuristic Model Checking for Java Programs,''  SPIN Workshop on Model Checking of Software, Grenoble, France, April 13, 2002.

  \item ``Adaptive Model Checking,''  International Conference on Tools and Algorithms for the Construction and Analysis of Systems, Grenoble, France, April 11, 2002.

  \item ``Structural Heuristics for Directed Model Checking of Java Programs,''  Specification and Verification Center, Carnegie Mellon University, Pittsburgh PA,  March 19, 2002.

  \item ``Efficient Model Checking Via B\"uchi Tableau Automata,''  International Conference on Computer Aided Verification, Paris, France, July 19, 2001.     

  \item ``Black Box Checking,''  Federal University of Rio Grande do Norte, Natal, Brazil, March 29, 2001.

\end{Selected Presentations}

\newpage

\begin{References}
Edmund M. Clarke, Jr. \\
FORE Systems Professor of Computer Science \\
Professor of Electrical and Computer Engineering\\
Computer Science Department \\
Carnegie Mellon University \\
5000 Forbes Avenue \\
Pittsburgh, PA 15213-3891 \\
412-268-2628\\
{\it emc+@cs.cmu.edu}

\and

Gerard J. Holzmann \\
JPL Fellow and Senior Research Scientist\\
Laboratory for Reliable Software\\
Jet Propulsion Laboratory\\
4800 Oak Grove Drive \\
M/S 301-230 \\
Pasadena, CA 91109 \\
818-393-5937 \\
{\it Gerard.J.Holzmann@jpl.nasa.gov}

\and

Willem Visser \\
Professor of Computer Science\\
Head, Computer Science Division\\
University of Stellenbosch\\
Private Bag X1\\
7602 Matieland\\
South Africa\\
+27 21 808 4235\\
{\it willem@gmail.com}
\end{References}
\end{vita}
\end{document}
