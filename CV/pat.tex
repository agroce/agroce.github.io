\documentclass[ComputerScience]{vita}
\usepackage{fullpage}
\usepackage{times}

\begin{document}
  \name{Patrick Francis Riley}
  \businessAddress{Carnegie Mellon University \\ Computer Science
    Department \\ 5000 Forbes Avenue \\ Pittsburgh, PA 15213-3891 \\
    (412)268-7123}
  \homeAddress{327 S. Negley Ave\\ Pittsburgh, PA 15232-1102
    \\ (412)665-1682}

\newcategory{Research Experience}
\newcategory{Teaching Experience}
\newcategory{Professional Activities}

\begin{vita}
  
  
  \begin{Degrees}
   \item B.S., Computer Science, Carnegie Mellon University,
    Pittsburgh, PA, May 1999.
   \item M.S., Computer Science, Carnegie Mellon University,
    Pittsburgh, PA, May 2002.
  \end{Degrees}


  \begin{Research Experience}
   \item 8/1999--present Ph.D. Graduate Student, Carnegie Mellon University.
    
    Studying mainly the coaching problem as an intermediate point
    between centralized and distributed control in multi-agent
    systems. Includes work with many areas such as: machine
    learning, behavior modelling, adaptive algorithms, and distributed
    planning. 
    % This is so out of date now!
%    Studied theory and specific problems for online algorithms,
%    with an emphasis on applying such algorithms to multi-agent games.
%    Also studied some game theory and various agent classification,
%    tracking, and adaptation techniques. Advised by Manuela Veloso and
%    Avrim Blum.

   \item 5/1999--8/1999 Research Assistant
    
    Developed a simulated robotic soccer team which was the champion
    of the international competition RoboCup-99. Also continued
    previous work on classifying adversary behaviors, focusing on
    using machine learning techniques.

   \item 8/1998--5/1999 Senior Undergraduate Research Project

    Designed and implemented a system for an agent team to classify
    an opponent by fusing partial information of the agents. Advised
    by Manuela Veloso.


   \item 1/1998--9/1998 Undergraduate Research Assistant

    Part of a three person team which designed and implemented a
    robotic soccer team which was the champion of the international
    competition RoboCup-98.

  \end{Research Experience}  

  
  \begin{Teaching Experience}
   \item Fall 2002; Teaching Assistant, Carnegie Mellon University,
    for the undergraduate course, \emph{Artificial Intelligence},
    covering the basic ideas of Artificial Intelligence research.
    Helped formulate assignments and tests, graded assignments and
    tests, and held weekly office hours.

   \item Spring 2000; Teaching Assistant, Carnegie Mellon University, for
    the undergraduate course, \emph{Great Theoretical Ideas in Computer
      Science}, a version of discrete math for Computer Science
    majors. Prepared and delivered a weekly recitation section, helped
    formulate assignments (including automatic grading programs) and
    tests, graded assignments and tests, and held weekly office hours.

   \item Fall 1998; Teaching Assistant, Carnegie Mellon University, for
    the undergraduate \emph{Artificial Intelligence}
    course. Prepared and delivered a lecture, formulated assignments and
    tests, graded assignments and exams, and held weekly office hours.
  \end{Teaching Experience}


  \begin{Honors}
   \item Best Paper Award at the Sixth International Conference on AI
    Planning and Scheduling (AIPS 2002)
   \item RoboCup Engineering Award at RoboCup 2002
   \item National Science Foundation Graduate Fellowship
   \item Allen Newell Award for Excellence in Undergraduate Research
   \item Andrew Carnegie Society Presidential Scholar
  %    \item Honorable Mention for Computing Research Association Outstanding Undergraduate Award
  %    \item Microsoft Technical Award
  \end{Honors}


  \begin{Professional Activities}
    \item Member RoboCup 2003 Simulation League Organizing Committee
    \item Member RoboCup 2002-2003 Simulation League Technical Committee
    \item Program Committee for Autonomous Agents and Multi-Agent
     Systems 2003
    \item Member RoboCup 2002 Simulation League Organizing Committee
  \end{Professional Activities}
  
  %Here's how this works: Just grab this info from a bbl file where
  %you print your stuff
  \begin{Publications}

  %\bibitem{LNAI02-mpades}
   \item 
    Patrick Riley.
    \newblock {MPADES}: Middleware for parallel agent discrete event simulation.
    \newblock In Gal~A. Kaminka, Pedro~U. Lima, and Raul Rojas, editors, {\em
      {R}obo{C}up-2002: The Fifth {R}obo{C}up Competitions and Conferences}.
    Springer Verlag, Berlin, 2003.
    \newblock (to appear).

  %\bibitem{LNAI02-coachable}
   \item 
    Paul Carpenter, Patrick Riley, Manuela Veloso, and Gal Kaminka.
    \newblock Integration of advice in an action-selection architecture.
    \newblock In Gal~A. Kaminka, Pedro~U. Lima, and Raul Rojas, editors, {\em
      {R}obo{C}up-2002: The Fifth {R}obo{C}up Competitions and Conferences}.
    Springer Verlag, Berlin, 2003.
    \newblock (to appear).

  %\bibitem{DARS02-empirical-coach}
   \item 
    Patrick Riley, Manuela Veloso, and Gal Kaminka.
    \newblock An empirical study of coaching.
    \newblock In H.~Asama, T.~Arai, T.~Fukuda, and T.~Hasegawa, editors, {\em
      Distributed Autonomous Robotic Systems 5}, pages 215--224. Springer-Verlag,
    2002.

  %\bibitem{AAMAS02-coachprob}
   \item
    Patrick Riley, Manuela Veloso, and Gal Kaminka.
    \newblock Towards any-team coaching in adversarial domains.
    \newblock In {\em Proceedings of the First International Joint Conference On
      Autonomous Agents and Multi-Agent Systems}, 2002.
    \newblock (poster paper).

  %\bibitem{AIPS02-STN}
   \item
    Patrick Riley and Manuela Veloso.
    \newblock Planning for distributed execution through use of probabilistic
    opponent models.
    \newblock In {\em Proceedings of the Sixth International Conference on AI
      Planning and Scheduling (AIPS-2002)}, 2002.

  %\bibitem{RCSymp2001-ATAC}
   \item
    Patrick Riley and Manuela Veloso.
    \newblock Recognizing probabilistic opponent movement models.
    \newblock In {\em Proceedings of the RoboCup International Symposium}, 2001.
    \newblock (poster paper).

  %\bibitem{Agents01-CoachingPoster}
   \item
    Patrick Riley and Manuela Veloso.
    \newblock Coaching a simulated soccer team by opponent model recognition.
    \newblock In {\em Proceedings of the Fifth International Conference on
      Autonomous Agents (Agents-2001)}, 2001.
    \newblock (extended abstract).

  %\bibitem{DARS-AdvClass}
   \item
    Patrick Riley and Manuela Veloso.
    \newblock On behavior classification in adversarial environments.
    \newblock In Lynne~E. Parker, George Bekey, and Jacob Barhen, editors, {\em
      Distributed Autonomous Robotic Systems 4}, pages 371--380. Springer-Verlag,
    2000.

  %\bibitem{IAAI-IdealModels}
   \item
    Peter Stone, Patrick Riley, and Manuela Veloso.
    \newblock Defining and using ideal teammate and opponent models.
    \newblock In {\em Proceedings of the Twelfth Innovative Applications of
      Artificial Intelligence Conference (IAAI-2000)}, 2000.

  %\bibitem{AAAI-StudentAbstract-AdvClass}
   \item
    Patrick Riley and Manuela Veloso.
    \newblock Towards behavior classification: A case study in robotic soccer.
    \newblock In {\em Proceedings of the Seventeenth National Conference on
      Artificial Intelligence (AAAI-2000)}, Menlo Park, CA, 2000. AAAI Press.
    \newblock (student abstract).

  %\bibitem{ATAL-LD}
   \item
    Patrick Riley, Peter Stone, and Manuela Veloso.
    \newblock Layered disclosure: Revealing agents' internals.
    \newblock In C.~Castelfranchi and Y.~Lesp\'{e}rance, editors, {\em Intelligent
      Agents VII. Agent Theories, Architectures, and Languages ---
      7th.~International Workshop, ATAL-2000, Boston, MA, USA, July 7--9, 2000,
      Proceedings}, Lecture Notes in Artificial Intelligence. Springer-Verlag,
    Berlin, Berlin, 2001.
    \newblock In this volume.

  %\bibitem{SeniorThesis}
   \item
    Patrick Riley.
    \newblock Classifying adversarial behaviors in a dynamic, inaccessible,
    multi-agent environment.
    \newblock Technical Report CMU-CS-99-175, Carnegie Mellon University, 1999.

  %\bibitem{LNAI99-simulator}
   \item
    Peter Stone, Patrick Riley, and Manuela Veloso.
    \newblock The {CMU}nited-99 champion simulator team.
    \newblock In Veloso, Pagello, and Kitano, editors, {\em {R}obo{C}up-99: Robot
      Soccer World Cup III}, pages 35--48. Springer, Berlin, 2000.

  %\bibitem{LNAI98-simulator}
   \item
    Peter Stone, Manuela Veloso, and Patrick Riley.
    \newblock The {CMU}nited-98 champion simulator team.
    \newblock In Asada and Kitano, editors, {\em {R}obo{C}up-98: Robot Soccer World
      Cup II}, pages 61--76. Springer, 1999.

    
  \end{Publications}
  

  \begin{References}
    Dr. Manuela Veloso\\
    Associate Professor of Computer Science \\
    Computer Science Department\\
    Carnegie Mellon University\\
    Pittsburgh  PA 15213-3891\\
    (412)268-1474\\

    \and

    Dr. Gal Kaminka \\
    Post Doctoral Fellow \\
    Computer Science Department\\
    Carnegie Mellon University\\
    Pittsburgh  PA 15213-3891\\

    \and

    Dr. Peter Stone \\
    Senior Technical Staff Member \\
    AT\&T Labs Research \\
    180 Park Ave \\
    Florham Park, NJ 07932 \\
    (973)360-8333
    

%    \and 

%    Avrim Blum \\
%    Computer Science Department\\
%    Carnegie Mellon University\\
%    Pittsburgh  PA 15213-3891\\
%    (412)268-6452\\

%    \and

%    Tucker Balch\\
%    The Robotics Institute\\
%    Carnegie Mellon University\\
%    5000 Forbes Avenue\\
%    Pittsburgh, PA 15213-3891\\
%    (412)268-1780
  \end{References}

\end{vita}

\end{document}

% Stuff to add
Systems, Man and Cybernetics - Part B  reviewer jan,feb 2003
JAIR reviewer, May, 2003
RC Symp Program Committee 2003
need to update references